\begin{abstract}
The performance of the code generated by a compiler depends on the order in which the optimization passes are applied.  In the context of high-level synthesis, the quality of the generated circuit relates directly to the code generated by the front-end compiler.
%The impact of compiler optimization is order-dependent.
%To produce good results, the compiler needs knowledge of the program features, the underlying systems and the run-time profiles. 
Unfortunately, choosing a good order--often referred to as the {\em phase-ordering} problem--is an NP-hard problem. As a result, existing solutions rely on a variety of sub-optimal heuristics.
%This NP-hard challenge is generally referred to as the phase-ordering challenge of the compiler.

In this paper, we evaluate a new technique to address the phase-ordering problem: deep reinforcement learning.
%The advances of deep reinforcement learning open a new horizon for addressing this challenge.
%In this paper, we explore the benefits of using deep reinforcement learning to achieve a better ordering of the compiler optimization passes.
To this end, we implement a framework that takes any group of programs and finds a sequence of passes that optimize the performance of these programs. 
%\JENNY{strictly speaking the sequence is not optimal}
% As a case study,
Without loss of generality, we instantiate this framework in the context of an LLVM compiler and target multiple High-Level Synthesis programs. 
%in theis used to target multiple High-level Synthesis programs with the LLVM compiler. 
We compare the performance of deep reinforcement learning to state-of-the-art algorithms that address the phase-ordering problem.
Overall, our framework runs one to two orders of magnitude faster than these algorithms, and achieves a 16\% improvement in circuit performance over the -O3 compiler flag.
\end{abstract}